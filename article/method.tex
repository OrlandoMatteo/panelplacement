\section{Methodology}
\label{sec:method}
evaluation of the irradiance traces for the roof all of the district

evaluation of the 75 percentile for the traces


Find all the possible position inside the available surface, sort them by the 75-th percentile, eliminate the panel with percentile lower than a threshold, find the configuration for the panels them taking into account distance between panels and height difference (description of the algorithm)

Given the best configuration try to place the same amount of panels on the used roof in a more classical way (classical algorithm description)

evaluate the roi given the panel costs the maintenance cost and the production

\hline
The goal of the paper is to find the best possible configuration for a PV panels system for a group of blocks of the city of Turin, considering the possibility to connect panel across contiguous roofs. The first step identifies the area of the blocks that could be exploited to install the PV panels. Then we proceed with the generation of the traces of G and T for the whole area with a fine time and space resolution (A). In the second step we evaluate a statistical measure to find which points of the suitable areas are the most illuminated during the year (B). The third step consist in the placement of the panel with a greedy approach and with a classical approach (C), then we evaluate the yearly production for bot the approaches (D).

\subsection{A}
Starting from a Digital Surface Model the algorithm identifies possible encumbrances an the evolution of the shadows to find the possible areas of the roof which could be used for the panel deployment. (Spiegazione di come funziona l'algoritmo che trova area_suit, forse lo avevamo fatto anche al corso di dottorato? @Lorenzo)
This step also allows us to know the inclination and the aspect of the roof that will be used in the following steps of the proceure.
The first part of the algorithm exploits high resolution maps of irradiance and temperature generated by the framework developed in [primo articolo]
Using the suitable areas indetified before the value of the irradiance over the time is evaluated by using together weather data and the shadow model. The spatial resolution of this traces is the same one of the DSM (i.e. 1 m) while the time resolution is 15 minutes. 

\subsection{B}
The following step consists in the calculation of the 75-th percentile for each point of this suitable areas, these values allows us to identify the more illuminated points to be used for the PV panel placement. 
The second part finds all the possible places where a single panel could be placed, the dimension of the panel to consider and the resolution of the search can be defined by the user. For each of this position we calculate a performance indicator as the minimum 75-th percentile of the points that intersects the panel surface, then we sort the list of those positions according to that value.
For this list of panels positions we defined different threshold values for the performance indicator to limit the number of positions to consider for the following steps.

\subsection{C}
The third part of the algorithm uses the list of panels posiiton fouund in the previous step to find an optimal panel configuration. 

