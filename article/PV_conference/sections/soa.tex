% citare gli altri che sembrano un po' out of topic per indicarli come diversi aspetti del placement dell'argomento.
The optimal placement of Distributed Energy Resources (DES) and the has a lot of aspcet that have been studied in literature. The combination of different type of DES has been investigated to optimize the efficiency of the grid \cite{masoum2010optimal}\cite{ondeck2015optimal}\cite{doagou2013optimal}. Another important aspect is the study of the placement of battery for the energy storage in order to exploit as much as possible the energy produced by the DES\cite{fortenbacher2017optimal}\cite{chedid2019optimal}.
The analysis of the solar potential and the panel placement are raising is importance due to the shifting towards the usage of PV panels as the most common technology among the Renewable DES. Hence the usage of GIS technologies is an useful and essential tools to simulate PV production in urban environment using Digital Surface Model (DSM) or 3D city models \cite{kucuksari2014integrated}\cite{pillot2020integrated}\cite{yushchenko2018gis}. The analysis of the solar potential of extended areas has been studied in \cite{baranyai2021correlation}\cite{khemiri2018optimal}\cite{bergamasco2011scalable}, in these works they analyze the potential of energy production of wide areas such entire is islands or region. These work similarly to our used GIS tools to extract irradiance information about the are of interest to than estimate which area ar the best for the panel installation. Their analysis however does not take into account the roof exploitable for the installation, it does not propose any method for the panel placement. The works proposed in \cite{kucuksari2014integrated}\cite{bracco2018energy} focus on the smaller scale of the district level to estimate the solar potential of the different roofs and calculate the expected energy production. However both these works estimates the energy production considering standard panels installation without taking advantage of fine grained information to maximize such production.
The work proposed in \cite{el2020optimal} has an approach closer to ours, it uses detailed historical data of the irradiance to evaluate the bests pv installation over the roofs of the houses a district. Unlike our approach however they use a classical installation method and their focus is to find the best placement for each single house thus not taking into account the possibility of connection among different roofs. That work however focuses on the needs of a single household instead of taking into consideration the figure of the EA, thus the possibility to group the production of multiple installation.
With respect to the literature this work proposes a framework to investigate relatively wide area of a urban context (1.7 $km^2$) to find different possible optimal configuration of PV panels. The framework exploits high resolution Digital Surface Models and historical weather data to identify the best positions to be used for the panels, considering the possibility to connect among each other panels located on contiguous roofs. Than we evaluate the Payback Time for each one of the resulting configuration in order to provide a tangible indicator for the Energy Aggregator that could be interested in such kind of analysis.