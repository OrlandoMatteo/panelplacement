The paper proposed a framework to support optimal installation of PV modules in a city district, with the goal of maximizing the profit for an EA. The approach is based on an efficient management of DSM data, that generates detailed irradiance traces only for the promising portion of the district roofs ($\sim0.5\%$ of total district area). The data is then used to build an optimal placement of PV modules, that can be parametrized to exclude positions affected by shading and by a discontinuous irradiance over time. The determined placement allows to find the suitable trade-off between initial investment, power production and payback time of the installation, and proved to generate a surplus power production of up to $+20\%$ w.r.t. a traditional installation. \attenzione{Future work?}