PhotoVoltaic (PV) installations are a widespread source of renewable energy, and are quite common urban buildings' roofs. To soften both the initial investment and the recurrent maintenance costs, the current market trends delegate the construction of PV installations to \emph{Energy Aggregators}, i.e., grouping of consumers and producers that act as a single entity to satisfy local energy demand and to sell the surplus energy to the grid. In this perspective, PV installations can be designed with a larger perspective, i.e., \emph{at distric level}, to maximize power production not of a single building but rather of a number of blocks of a city. This implies new challenges, including efficient data management (the covered area can be squared kilometers wide) and optimal PV installation (the number of PV modules can be in the order of hundreds or even thousands). This paper proposes a framework to combine detailed geographic and irradiance information to determine an \emph{optimal PV installation over a district, by maximizing both power production and economic convenience}. Our simulation results on a real-world district demonstrate an improvement on power generation up to 20\%  w.r.t. a standard compact installations, and that the framework allows an advanced evaluation of costs and benefit that can be used by EA to design a new PV installation. 