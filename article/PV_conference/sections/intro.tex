% Inserire riferimenti all'articolo dell'anione europea linkato da sara su skype 
% Positive energy district, comunita energetica, zero energy district, zero energy building (forse zero energy e piu legato al termico, controllare)
% Verificare se Compsac double blind

%The challenges posed by climate change are currently promoting the adoption of renewable energy sources (RES) instead of fossil fuel energy generation, to the point that projections estimate that RESs will provide 60\% of total energy consumption by 2050 and more than 60\% of the newly installed global electricity capacity by 2040 
%\cite{baranyai2021correlation}.
Among the various Renewable Energy Sources (RES), PhotoVoltaic (PV) energy generation is one of the most interesting solutions, as an effect of its increasing efficiency, reduced cost and easiness of installation, with an estimated market share of 25\% of power generation achieved through PV installations by 2050 \cite{irena}. %, with estimations that supplying 25% of total electricity demand by 2050 https://irena.org/-/media/Files/IRENA/Agency/Publication/2019/Nov/IRENA_Future_of_Solar_PV_2019.pdf
% The reduction of $CO_2$ emission to contrast the global warming is among the main objective for the societies \cite{COP21}. This necessity is promoting the renewable energy sources (RES) to replace the generation of energy through fossil fuel. Among the RES the Photovoltaic is probably the most common technology due to its reduced cost and the easiness of installation. 

The adoption of PV installations is currently encouraged also by the novel market \emph{prosumer} paradigm: energy consumers become also producers, as their residential RES installations not only meet user demand, but generate a potential surplus production that can be sold to the energy grid \cite{su11195286}. 
% Moreover in the upcoming context of smart grid and micro-grid the prosumer (producer and user) would benefit a lot more from the PV systems as the would be able to sell to other user or to the grid the energy in excess. 
While being a promising solution, applying the prosumer paradigm to a single household is not always a viable solution: householders may not affort the cost of installation and maintenance of a PV installation, or may not be willing to make a financial investment in light of possible future earnings. % householders may not afford the investment needed for the installation and maintenance of PV panels, even in a smart grid scenario the earnings due to the energy sold to the grid may be risible and they would not be interesting for the householder with the financial possibilities. 

To overcome this problem, the current market solutions operate at the district level, where a number of buildings cooperate to constitute a larger PV installation and an \emph{Energy Aggregator} (EA) aggregates the overall energy demand and takes care of selling the surplus production to the grid \cite{lu2020fundamentals}. In this way, single prosumers do not need to care about the investment and the management of the PV systems, still achieving the advantages of potential energy independence \cite{mizzimi_2020}.
% by introducing the figure of the Energy Aggregator (EA) is raising in importance. The EA is an actor that stays in between the grid and the prosumer by aggregating the energy production a lot of them and by selling it to the grid \cite{lu2020fundamentals}. This figure helps both the grid, as it ensure a more stable and significant amount of energy to buy, and the prosumer who does not need to care about the investment an the management of the PV systems \cite{mizzimi_2020}.

To fully benefit from the new market paradigms, the EA must carefully design the PV installation in the area of interest, so to fully exploit its solar potential. Buildings indeed project shadows that generate heavy partial-shading effects, thus reducing the efficiency of PV power generation and requiring a careful trade-off between the size of installation (with the consequent costs) and the return of investment generated by power generation \cite{ZHU2019831}: it is often the case that a larger PV installation does not lead to larger earnings, as an effect of a larger initial investment and of an ineffective power production in a portion of the installation area, subject to shading effects.
%This raises thus a number of challenges, such as: the identification of the most suitable district for PV deployment \cite{quiros2018solar}, finding the best combination of PV systems with other energy sources \cite{ondeck2015optimal}, the identification of optimal connection to the grid and of the optimal battery configuration \cite{8257681,jannesar2018optimal}.
% The EA however need to have a well studied plan to fully exploit the solar potential of the roofs in a urban context, in some cases placing a lot of panels may not be cost-effective as the initial investment would be payed back in too much time or the gain in energy production would not be worth the additional money.
%However the presence of the figure of the Energy aggregator raised various challenges to face in order to reach positive financial result while preserving the stability of the network. There are different aspect of these challenges that have been analyzed: the identification of the right zone of a city to be used for the deployment of the panels \cite{quiros2018solar}, the combination of PV systems with other energy sources\cite{ondeck2015optimal}, the identification of optimal connection of the installation to the grid to ensure the efficiency and the stability \cite{8257681}, the placement of battery to store the energy for future use\cite{jannesar2018optimal}.

In this scenario, identifying the most suitable roofs of a district to achieve optimal PV power generation and determining the corresponding optimal PV installation is a relevant problem. Not only the problem is complex, but it also requires different skills, ranging from shadow forecast, to PV power generation and optimization, and economic estimation of the return of investment. 
This work proposes solution to such a complex scenario with \emph{a framework that works at district level to determine the optimal PV installation} from the perspective of both costs/benefit trade-off and of production efficiency.  

The novelty of this work lies in the following contributions: 
% defines a framework to identify and analize multiple optimal PV installations in terms of power production and Payback Time (PT). The framework proposed here does not simply consist in analyze each single roof by itself with the strategy proposed in \cite{compsac2020}. Instead, by 
\begin{itemize}
    \item a GIS-based approach is used to evaluate the evolution of irradiance and temperature over the roofs of a district over one year, by achieving a good spatial resolution ($1m$) to allow an accurate estimation of the operating conditions of a possible PV installation;
    \item the identification of the optimal placements of PV panels over the district, achieved by considering the roofs of district as a whole, i.e., allowing to connect panels located on contiguous roofs of different buildings; 
    \item an economic analysis to determine the payback time of the PV installation; 
    \item a trade-off analysis that considers the payback time and the return of investment of the installation, by considering different sizes of the PV installation and allowing different levels of PV efficiency, to determine the most suitable and the most economically convenient solution in the interest of the EA; 
    \item the application to a district located in \omitted{Turin, Italy}, will prove an improvement of power production of up to 20\% and of 25\% of payback time w.r.t. a traditional installation. 
\end{itemize}
 %To achieve such result a GIS-based approach is used to evaluate the evolution of irradiance and temperature over the roofs of a group of city block over one year. This kind of detailed data allows to determine the best areas to be used for the PV panels. 
%Using this data we defined different minimum threshold for the irradiance value to evaluate different possible PV installation and we compared it with a classical approach for the panel placement to verify the performance of the proposed framework. Results showed and improvement on the power generation from 2\% upt o 20\%.
%For each of this installation we evaluated both the energy production and the PT in order to help a possible EA to identify with solution fits best his possibility and his needs.

The paper is organized as follows. Section \ref{sec:soa} reviews relevant literature solutions, and section \ref{sec:back_pv} provides the necessary background on PV power generation. Section IV presents the proposed methodology. Section
V discusses the experimental results. Finally, Section VI provides our concluding remarks.